\documentclass[aps,prd,nofootinbib,twocolumn,reprint,superscriptaddress,showpacs,showkeys,longbibliography]{revtex4-1}
\pdfoutput=1
\usepackage {amsmath}
\usepackage {amssymb}
\usepackage {amsfonts}
\usepackage {amsthm}
\usepackage {mathrsfs}
\usepackage {natbib}
\usepackage {latexsym}
\usepackage {graphicx}
\usepackage {dsfont}
\usepackage {txfonts}
\usepackage {rotating}
\usepackage {wasysym}
\usepackage {multirow}
\usepackage {hhline}
\usepackage {graphicx}
\usepackage {hyperref}
\usepackage[dvipsnames,usenames]{color}
\usepackage {bm}
\usepackage{appendix}
\usepackage{acronym}
\usepackage{dcolumn}   % needed for some tables
\usepackage{enumitem}
\usepackage{bigdelim}
\usepackage {url}
\usepackage{caption}
\usepackage {subcaption}
\usepackage{multirow}
% REMINDER TO CHRIS: subcaption.sty might need to be included,
% or else the package removed; isn't universally installed -- GDM

\usepackage{framed}

%% ----- macros lifted from aastex.cls
\newcommand\arcdeg{\mbox{$^\circ$}}%
\newcommand\arcmin{\mbox{$^\prime$}}%
\newcommand{\bra}[1]{\langle #1|}
\newcommand{\ket}[1]{|#1\rangle}
\newcommand{\braket}[2]{\langle #1|#2\rangle}

%% ----- some handy shortcuts
\newcommand{\dcc}{LIGO-XXXXXXXXX}
%% -----
\newcommand{\cm}[1]{\textbf{\textcolor{red}{CM: #1}}}
\newcommand{\dg}[1]{\textbf{\textcolor{blue}{DG: #1}}}
%% ----- Version that makes the comments disappear
%% -----
\newcommand\marginnote[1]{%
    \mbox{}\marginpar{\raggedleft\hspace{0pt}\footnotesize\textcolor{green}
\raggedright\hspace{0pt}\footnotesize\textcolor{green}
    {#1}}}
%% ----- macros for cross-references

%% ----- input git-version tag
\input{tag.tex}

%% ----- define shorthand variables

\begin{document}

\title{Quantum matched filtering}

% NOTE TO EVERYBODY: we should decide on whether to make
% names full or initialized -- GDM
\author{Sijia Gao}
\email{s.gao.2@research.gla.ac.uk}
\affiliation{SUPA, School of Physics and Astronomy, University of Glasgow, Glasgow G12 8QQ, United Kingdom}
\author{Fergus Hayes}
\email{f.hayes.1@research.gla.ac.uk}
\affiliation{SUPA, School of Physics and Astronomy, University of Glasgow, Glasgow G12 8QQ, United Kingdom}
\author{Sarah Croke}
\affiliation{SUPA, School of Physics and Astronomy, University of Glasgow, Glasgow G12 8QQ, United Kingdom}
\author{J.~Veitch}
\affiliation{SUPA, School of Physics and Astronomy, University of Glasgow, Glasgow G12 8QQ, United Kingdom}
\author{C.~Messenger}
\affiliation{SUPA, School of Physics and Astronomy, University of Glasgow, Glasgow G12 8QQ, United Kingdom}

\date{\today}
\date{\commitDATE\\\mbox{\small{\commitID} \commitSTATUS}\\\mbox{\dcc}}

\begin{abstract}
We did a quantum thing
\end{abstract} 

\maketitle

\acrodef{NS}[NS]{Neutron Star}
\acrodef{GW}[GW]{gravitational-wave}

%%%%%%%%%%%%%%%%%%%%%%%%%%%%%%%%%%%%%%%%%%%%%%%%
%%%%%%%%%%%%%%%%%%%%%%%%%%%%%%%%%%%%%%%%%%%%%%%%
\section{Introduction}\label{sec:intro}

\begin{itemize}
\item GW introduction
\item Quantum computing introduction
\item summary of what this paper intends to do 
\end{itemize}

%%%%%%%%%%%%%%%%%%%%%%%%%%%%%%%%%%%%%%%%%%%%%%%%
%%%%%%%%%%%%%%%%%%%%%%%%%%%%%%%%%%%%%%%%%%%%%%%%
\section{Background}\label{sec:background}

\subsection{Matched-filtering}

\begin{itemize}
\item derive matched-filtering as a semi-optimal statistic
\item What problem it solves
\end{itemize}

\subsection{Grovers algorithm}

\begin{itemize}
\item introduce Grover's algorithm
\item What problem it solves
\item Why are we using Grovers - the largest parameter is the number of
templates (not the dataset size)
\item introduce the notion of the Oracle
\item introduce phase kick-back
\item quantum counting - phase estimation and qunatum Fourier transforms
\end{itemize}

%%%%%%%%%%%%%%%%%%%%%%%%%%%%%%%%%%%%%%%%%%%%%%%%
%%%%%%%%%%%%%%%%%%%%%%%%%%%%%%%%%%%%%%%%%%%%%%%%
\section{Quantum psuedo-code}\label{sec:psuedocode}

\begin{itemize}
\item How to construct the Oracle
\item A counting argument for the complexity - templates are indices
\item discuss template erasure
\item actually show the psuedo-code representation
\end{itemize}

%%%%%%%%%%%%%%%%%%%%%%%%%%%%%%%%%%%%%%%%%%%%%%%%
%%%%%%%%%%%%%%%%%%%%%%%%%%%%%%%%%%%%%%%%%%%%%%%%
\section{Example using Qiskit}\label{sec:qizkitexample}

\begin{itemize}
\item define model (small number of qubits and exact matching)
\item focus on results
\end{itemize}

%%%%%%%%%%%%%%%%%%%%%%%%%%%%%%%%%%%%%%%%%%%%%%%%
%%%%%%%%%%%%%%%%%%%%%%%%%%%%%%%%%%%%%%%%%%%%%%%%
\section{Example on Sine wave (python)}\label{sec:sineexample}

\begin{itemize}
\item continuous wave GW signal motivation
\item why we don't just use the QFT? Our real CW signals (and CBCs) are not monochromatic
\item Define sine wave model
\item focus on showing results - number of steps taken
\end{itemize}

%%%%%%%%%%%%%%%%%%%%%%%%%%%%%%%%%%%%%%%%%%%%%%%%
%%%%%%%%%%%%%%%%%%%%%%%%%%%%%%%%%%%%%%%%%%%%%%%%
\section{CBC application example}\label{sec:cbcexample}

\begin{itemize}
\item define the CBC waveform model
\item focus on results - compare to GWOSC GW150914 data
\item make the number of steps clear (~ root N operations)
\end{itemize}

%%%%%%%%%%%%%%%%%%%%%%%%%%%%%%%%%%%%%%%%%%%%%%%%
%%%%%%%%%%%%%%%%%%%%%%%%%%%%%%%%%%%%%%%%%%%%%%%%
\section{Practical considerations}\label{sec:intro}

Discuss things about practicalities for the future. How many qubits are needed
etc.. 

\begin{itemize}
\item space requirements on quantum computer
\item describe the current state of the art in quantum processors
\end{itemize}

%%%%%%%%%%%%%%%%%%%%%%%%%%%%%%%%%%%%%%%%%%%%%%%%
%%%%%%%%%%%%%%%%%%%%%%%%%%%%%%%%%%%%%%%%%%%%%%%%
\section{Discussion}\label{sec:discussion}

\begin{itemize}
\item Summarise the entire paper first
\item highlight the parts of interest to Quantum and GW fields separately
\item Discuss limitations of what we've done
\item Talk about future work
\item Talk about the future of Quantum computers and the future GW data
analysis challenges
\item It's not just GWs though - data analysis searches in general
\end{itemize}

%%%%%%%%%%%%%%%%%%%%%%%%%%%%%%%%%%%%%%%%%%%%%%%%
\begin{acknowledgments}
hello
\end{acknowledgments}


%%%%%%%%%%%%%%%%%%%%%%%%%%%%%%%%%%%%%%%%%%%%%%%%
\appendix

%%%%%%%%%%%%%%%%%%%%%%%%%%%%%%%%%%%%%%%%%%%%%%%%
%%%%%%%%%%%%%%%%%%%%%%%%%%%%%%%%%%%%%%%%%%%%%%%%
\section{Complete MDC results\label{sec:fullresults}}


% Create the reference section using BibTeX:
\bibliography{masterbib}

\end{document}
